\documentclass{beamer}
\usetheme{Warsaw}
\setbeamertemplate{footline}[frame number]

\usepackage[utf8]{inputenc}
\usepackage{fancybox}
\usepackage{multimedia} 
\usepackage{subfig}
\usepackage{amsmath}
\usepackage{hyperref}
\usepackage[all]{xy}
\begin{document}


\title[Angewandte Mathematik] % (optional, only for long titles)
{Angewandte Mathematik
\\
\includegraphics[scale=0.15]{images/cover}
}
\subtitle{}
\author[Dr. Johannes Riesterer] % (optional, for multiple authors)
{Dr.  rer. nat. Johannes Riesterer}

\date[KPT 2004] % (optional)
{}

\subject{Angewandte Mathematik}

\frame{\titlepage}


\begin{frame}
\framesubtitle{Algorithmus}
    \begin{block}{Algorithmus}
\begin{figure}[H]
      \centering
    \includegraphics[width=0.7\textwidth]{images/algo}
      \caption{Quelle: Wikipedia}
\end{figure}
\end{block}

 \end{frame}




\begin{frame}
    \frametitle{Angewandte Mathematik}
\framesubtitle{}
    \begin{block}{Algorithmus Informell}
Ein Algorithmus ist eine eindeutige Handlungsvorschrift zur Lösung eines Problems oder einer Klasse von Problemen. Algorithmen bestehen aus endlich vielen, wohldefinierten Einzelschritten.
\end{block}
    \begin{block}{Algorithmus Formal}
\includegraphics[scale=0.8]{images/Turingmaschine}
\end{block}
 \end{frame}

\begin{frame}
\framesubtitle{Algorithmus}
    \begin{block}{Algorithmus}
\begin{figure}[H]
      \centering
    \includegraphics[width=0.7\textwidth]{images/computer}
      \caption{Quelle: Wikipedia}
\end{figure}
\end{block}

 \end{frame}






\begin{frame}
    \frametitle{Angewandte Mathematik}
\framesubtitle{Fehleranalyse}
    \begin{block}{Gleitkommazahl}
Eine Gleitkommazahl ist eine Zahl $z$ der Form
\begin{align*}
z = a d^e ; \;\;
a = (\pm) \sum_{i=1}^l c_i d^{-i} \\
e, c_i \in \{e_{min}, \cdots , e_{max}  \} \subset \mathbb{Z}
\end{align*}
\end{block}

    \begin{block}{Gleitkommazahl $d=10$}
\begin{align*}
0.314156 \cdot 10^1
\end{align*}
\end{block}

\begin{block}{Gleitkommazahl Darstellung $d=2$}
\begin{figure}[H]
    \centering
  \includegraphics[width=0.7\textwidth]{images/float}\end{figure}
\end{block}

 \end{frame}

 \begin{frame}
    \framesubtitle{Algorithmus}
        \begin{block}{Schaltwerke}
    \begin{figure}[H]
          \centering
        \includegraphics[width=0.7\textwidth]{images/Volladdierer}
          \caption{Quelle: Wikipedia}
    \end{figure}
    \end{block}
    
     \end{frame}
    

\begin{frame}
    \frametitle{Angewandte Mathematik}
\framesubtitle{Fehleranalyse}
    \begin{block}{Gleitkommazahl}
Ist x eine reelle Zahl so gibt es eine  Gleitkommazahl $fl(x)$ mit
\begin{align*}
\frac{|x-fl(x)| }{|x|} \leq eps := d^{1-l}/2
\end{align*}
\end{block}

 \end{frame}



\begin{frame}
    \frametitle{Angewandte Mathematik}
\framesubtitle{Fehleranalyse}
    \begin{block}{Gleitkommazahl}
Für eine exakte Operation $\circ \in \{+,-, \cdot, : \}$ gilt für die entsprechende Ausführung $\hat{\circ}$ auf einem Computer
\begin{align*}
a \hat{\circ}  b = (a \circ b) (1  + \epsilon) , \ \epsilon \leq eps 
\end{align*}
\end{block}
 \end{frame}


 \begin{frame}
    \frametitle{Angewandte Mathematik}
\framesubtitle{Fehleranalyse}
\begin{figure}[H]
      \centering
    \includegraphics[width=0.7\textwidth]{images/fehler}\end{figure}
 \end{frame}

\begin{frame}
    \frametitle{Angewandte Mathematik}
\framesubtitle{Fehleranalyse}
    \begin{block}{Konditionszahl}
 Die Kondition beschreibt  die Abhängigkeit der Lösung eines Problems von der Störung der Eingangsdaten.  Die Konditionszahl stellt ein Maß für diese Abhängigkeit dar. Sie beschreibt das Verhältnis von  $E:= \{\widetilde{x} \; | \; ||\widetilde{x} -x || \leq eps ||x|| \}$ zu $R: = \{f(\widetilde{x}) \; | \; \widetilde{x} \in E \}$.
\end{block}
\begin{figure}[H]
      \centering
    \includegraphics[width=0.8\textwidth]{images/kondition}
      \caption{}
\end{figure}

 \end{frame}



\begin{frame}
    \frametitle{Angewandte Mathematik}
\framesubtitle{Fehleranalyse}
    \begin{block}{Kondition eines Problems}
Die absolute Konditionierung eines Problems $(f,x)$ ist die Kleinste Zahl $\kappa_{abs}$ mit 
\begin{align*}
|| f(x) - f(\widetilde{x}) || \leq \kappa_{abs} || x - \widetilde{x} || , \;  \widetilde{x} \to x
\end{align*}
\end{block}

    \begin{block}{Kondition eines Problems}
Die relative  Konditionierung eines Problems $(f,x)$ ist die Kleinste Zahl $\kappa_{rel}$ mit 
\begin{align*}
\frac{|| f(x) - f(\widetilde{x}) ||}{||f(x) || } \leq \kappa_{rel} \frac{|| x - \widetilde{x} ||}{||x||} , \; \widetilde{x} \to x
\end{align*}
\end{block}

 \end{frame}


\begin{frame}
    \frametitle{Angewandte Mathematik}
\framesubtitle{Fehleranalyse}
    \begin{block}{Kondition eines Problems}
Momentan können wir noch keine Konditionszahlen berechnen. Wir werden später lernen, wie wir sie in vielen Fällen abschätzen können.
\end{block}
 \end{frame}




\begin{frame}
    \frametitle{Angewandte Mathematik}
\framesubtitle{Fehleranalyse}
    \begin{block}{Stabilität}
\end{block}
\begin{figure}[H]
      \centering
    \includegraphics[width=0.8\textwidth]{images/stabilitaet}
      \caption{}
\end{figure}
 \end{frame}



\begin{frame}
    \frametitle{Angewandte Mathematik}
\framesubtitle{Fehleranalyse}
    \begin{block}{Stabilität}
Für eine Gleikommarealisierung $\hat{f}$ eines Algorithmus zur Lösung des Problems $(f,x)$ mit relativer Konditionszahl $\kappa_rel$ ist der Stabilitätsindikator definiert als die kleinste Zahl $\sigma \geq 0$ mit 
\begin{align*}
\frac{|| \hat{f}(\widetilde{x}) - f(\widetilde{x}) ||}{||f(\widetilde{x}) || } \leq \sigma  \kappa_{rel} eps , \; eps \to 0
\end{align*}
für alle $\widetilde{x} \in E$
\end{block}
    \begin{block}{Stabilität eines Algorithmus}
Der Algorithmus $\hat{f}$ heisst stabil, wenn $\sigma$ kleiner ist als die Anzahl der hintereinander ausgeführten Elementaroperationen. 
\end{block}
 \end{frame}

\end{document}

