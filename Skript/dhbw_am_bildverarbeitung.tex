\documentclass{beamer}
\usetheme{Warsaw}

\usepackage[utf8]{inputenc}
\usepackage{fancybox}
\usepackage{multimedia} 
\usepackage{subfig}
\usepackage{amsmath}
\usepackage{hyperref}
\usepackage[all]{xy}
\begin{document}


\title[Digitale Bildverarbeitung] % (optional, only for long titles)
{Digitale Bildverarbeitung

}
\subtitle{}
\author[Dr. Johannes Riesterer] % (optional, for multiple authors)
{Dr.  rer. nat. Johannes Riesterer}

\date[KPT 2004] % (optional)
{}

\subject{Digitale Bildverarbeitung}

\frame{\titlepage}


\begin{frame}
    \frametitle{Digitale Bilder}
\framesubtitle{}
    \begin{block}{Umwandlung von Bildern}
\begin{itemize}
\item Viele Verfahren der  Signalverarbeitung haben ihren Ursprung in der Analysis. Um diese anwenden zu können, müssen diskrete Daten  in kontinuierliche Daten umgewandelt werden.
\item Auf der anderen Seite kann  ein Computer nur diskrete Daten  verarbeitet. Kontinuierliche Signale (zum Beispiel von Sensoren) müssen daher in diskrete Daten umgewandelt werden.
\end{itemize}
\end{block}

 \end{frame}

\begin{frame}
    \frametitle{Digitale Bilder}
\framesubtitle{}
    \begin{block}{}
Für ein eindimensionales, diskretes Bild $U : [1, \ldots, N]  \to R$ bezeichne  $U_{j} := U(j)$.
\end{block}
    \begin{block}{Stückweise konstante Interpolation}
Definiere $\phi^0 (x) : = 1_{ [-\frac{1} {2} ,  \frac{1} {2}) }(x) : =\begin{cases}
   1 , & \text{for } -\frac{1} {2}  \leq x < \frac{1} {2}  \\
    0  & \text{else } 
  \end{cases} $, $\phi^0_j(x):= \phi^0 (x - j)$ und
 $u(x) := \sum_{j=1}^{N} U_j \phi^0_j(x)$ 
\end{block}

 \end{frame}





\begin{frame}
    \frametitle{Digitale Bilder}
\framesubtitle{}
\begin{block}{Höherdimensionale stückweise  Interpolation}
Für ein 2-dimensionales, diskretes Bild $U : [1, \ldots, N] \times   [1, \ldots, M] \to R$ definiere 
 $u(x, y) := \sum_{i=1}^{N} \sum_{j=1}^{M}  U_{i,j} \cdot \phi_i(x) \cdot \phi_j(y)$ und analog für n-dimensonale Bilder....
\end{block}
 \end{frame}


\begin{frame}
    \frametitle{Digitale Bilder}
\framesubtitle{}
\begin{block}{Abtastung}
Für ein kontinuierliches  Bild $u : I^n \to R$ erhält man durch gewichtete Mittelungen
$U_i := \int_{I^n} \phi (x - x_i) u(x) dx$ ein diskretes Bild. 
\end{block}
 \end{frame}


\begin{frame}
    \frametitle{Integration}
\framesubtitle{}

    \begin{block}{Faltung}
\begin{align}
(f * g )(x) := \int_{\mathbb{R}^n}  f(y-x) \cdot g(y) \; dy 
\end{align}

\end{block}
    \begin{block}{Beispiel 1}
\href{https://moodle.dhbw-mannheim.de/pluginfile.php/278535/mod_folder/content/0/Convolution_of_box_signal_with_itself.gif?forcedownload=1}{Link: Box}
\end{block}
 
    \begin{block}{Beispiel 2}
\href{https://moodle.dhbw-mannheim.de/pluginfile.php/278535/mod_folder/content/0/Convolution_Animation_(Gaussian).gif?forcedownload=1}{Link: Gauß}
\end{block}
 
\end{frame}


\begin{frame}
    \frametitle{Diskrete Faltung}
\framesubtitle{}

    \begin{block}{Diskrete Faltung}
Für zwei diskrete  Funktionen $U : [1, \ldots, N]  \to R$ und $H : [1, \ldots, N]  \to R$ mit stückweisen konstanten Interpolation $u(x) := \sum_{l=1}^{N} U_l \phi^0_j(x)$ und 
$h(x) := \sum_{m=1}^{N} H_m \phi^0_m(x)$ ergibt die Faltung 
\begin{align*}
& (h * u)(k) = \int u(y)h(k-y)  \; dy \\
& = \int \ \sum_{l=1}^{N} U_l \phi^0(y-l) \sum_{m=1}^{N} H_m \phi^0(k-y-m) \\
& = \sum_{l=1}^{N}   \sum_{m=1}^{N} U_l  H_m  \int  \phi^0(y-l) \phi^0(k-y-m) \; dy 
\end{align*}

\end{block}
 \end{frame}

\begin{frame}
    \frametitle{Diskrete Faltung}
\framesubtitle{}

    \begin{block}{Diskrete Faltung}
Da für das Integral 
\begin{align*}
  \int  \phi^0(y-l) \phi^0(k-y-m)  \; dy  = \begin{cases}
1 \text{ falls } m = k -l\\
0 \text{ sonst }
\end{cases}
\end{align*}
gilt, folgt die Darstellung
\begin{align*}
 (u  * h)(k) = \sum_l U_l H_{k-l}
\end{align*}

\end{block}

 \end{frame}




\end{document}
