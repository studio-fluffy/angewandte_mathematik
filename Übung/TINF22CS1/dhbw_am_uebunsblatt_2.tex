\documentclass[a4paper,13pt]{scrartcl}

\usepackage[utf8]{inputenc}
\usepackage{amsfonts}
\usepackage{amsmath}
\usepackage{amssymb}
\usepackage{amsthm}
\usepackage{color}
\usepackage[ngerman]{babel}
\usepackage[pdftex]{graphicx}
%\usepackage[T1]{fontenc}
\usepackage{graphicx}
\pagestyle{empty}

%\topmargin20mm
\oddsidemargin0mm
\parindent0mm
\parskip2mm
\textheight25.5cm
\textwidth15.8cm
\unitlength1mm



\begin{document}
\section*{\large  Übungsblatt 2 \hfill Angewandte Mathematik }
\hrule
\hrule
\vspace{4mm}
%\includegraphics[width=0.8\textwidth]{sampleplot.pdf}

{\bf Aufgabe 1}
Betrachte die Menge  $M: = \{ \psi : I_\delta (t_0) \to \mathbb{R}^n \; | \; ||\psi(t) - x_0 || \leq b  \}$ von Wegen in der Nähe von $x_0$ und die Abbildung
\begin{align*}
& P : M \to M \\
& (P \psi)(t) := x_0 + \int_{t_0}^{t} F(t, \psi(t)) dt
\end{align*}
Zeigen Sie, dass ein Fixpunkt $\psi^*$ von $P$ eine Lösung der Differentialgleichung 
$\psi'(t) = F(t, \psi(t))$ ist.
\vspace{8mm}


{\bf Aufgabe 2}
Seien  $\varphi_1, \cdots, \varphi_n$ 
Lösungen der homogenen Gleichung $\varphi'(t) = A \varphi(t)$.
Zeigen Sie, dass dann
 $c_1 \cdot \varphi_1 + \cdots + c_n \cdot \varphi_n$  auch eine Lösung ist.
\vspace{8mm}

\vspace{8mm}

{\bf Aufgabe 3}

Programmieren Sie das Runge-Kutta Verfahren in Python. Lösen Sie damit Näherungsweise
das Räuber-Beute Model und plotten Sie die Lösungen in Python. 


\end{document}

